% --- LaTeX Lecture Notes Template - S. Venkatraman ---

% --- Set document class and font size ---

\documentclass[letterpaper, 12pt]{article}

% --- Package imports ---

% Extended set of colors
\usepackage[dvipsnames]{xcolor}

\usepackage{
  amsmath, amsthm, amssymb, mathtools, dsfont, units,          % Math typesetting
  graphicx, wrapfig, subfig, float,                            % Figures and graphics formatting
  listings, color, inconsolata, pythonhighlight,               % Code formatting
  fancyhdr, sectsty, hyperref, enumerate, enumitem, framed }   % Headers/footers, section fonts, links, lists

% lipsum is just for generating placeholder text and can be removed
\usepackage{hyperref, lipsum} 

% --- Fonts ---

\usepackage{newpxtext, newpxmath, inconsolata}

% --- Page layout settings ---

% Set page margins
\usepackage[left=1.35in, right=1.35in, top=1.0in, bottom=.9in, headsep=.2in, footskip=0.35in]{geometry}

% Anchor footnotes to the bottom of the page
\usepackage[bottom]{footmisc}

% Set line spacing
\renewcommand{\baselinestretch}{1.2}

% Set spacing between paragraphs
\setlength{\parskip}{1.3mm}

% Allow multi-line equations to break onto the next page
\allowdisplaybreaks

% --- Page formatting settings ---

% Set image captions to be italicized
\usepackage[font={it,footnotesize}]{caption}

% Set link colors for labeled items (blue), citations (red), URLs (orange)
\hypersetup{colorlinks=true, linkcolor=RoyalBlue, citecolor=RedOrange, urlcolor=ForestGreen}

% Set font size for section titles (\large) and subtitles (\normalsize) 
\usepackage{titlesec}
\titleformat{\section}{\large\bfseries}{{\fontsize{19}{19}\selectfont\textreferencemark}\;\; }{0em}{}
\titleformat{\subsection}{\normalsize\bfseries\selectfont}{\thesubsection\;\;\;}{0em}{}

% Enumerated/bulleted lists: make numbers/bullets flush left
%\setlist[enumerate]{wide=2pt, leftmargin=16pt, labelwidth=0pt}
\setlist[itemize]{wide=0pt, leftmargin=16pt, labelwidth=10pt, align=left}

% --- Table of contents settings ---

\usepackage[subfigure]{tocloft}

% Reduce spacing between sections in table of contents
\setlength{\cftbeforesecskip}{.9ex}

% Remove indentation for sections
\cftsetindents{section}{0em}{0em}

% Set font size (\large) for table of contents title
\renewcommand{\cfttoctitlefont}{\large\bfseries}

% Remove numbers/bullets from section titles in table of contents
\makeatletter
\renewcommand{\cftsecpresnum}{\begin{lrbox}{\@tempboxa}}
\renewcommand{\cftsecaftersnum}{\end{lrbox}}
\makeatother

% --- Set path for images ---

\graphicspath{{Images/}{../Images/}}

% --- Math/Statistics commands ---

% Add a reference number to a single line of a multi-line equation
% Usage: "\numberthis\label{labelNameHere}" in an align or gather environment
\newcommand\numberthis{\addtocounter{equation}{1}\tag{\theequation}}

% Shortcut for bold text in math mode, e.g. $\b{X}$
\let\b\mathbf

% Shortcut for bold Greek letters, e.g. $\bg{\beta}$
\let\bg\boldsymbol

% Shortcut for calligraphic script, e.g. %\mc{M}$
\let\mc\mathcal

% \mathscr{(letter here)} is sometimes used to denote vector spaces
\usepackage[mathscr]{euscript}

% Convergence: right arrow with optional text on top
% E.g. $\converge[p]$ for converges in probability
\newcommand{\converge}[1][]{\xrightarrow{#1}}

% Weak convergence: harpoon symbol with optional text on top
% E.g. $\wconverge[n\to\infty]$
\newcommand{\wconverge}[1][]{\stackrel{#1}{\rightharpoonup}}

% Equality: equals sign with optional text on top
% E.g. $X \equals[d] Y$ for equality in distribution
\newcommand{\equals}[1][]{\stackrel{\smash{#1}}{=}}

% Normal distribution: arguments are the mean and variance
% E.g. $\normal{\mu}{\sigma}$
\newcommand{\normal}[2]{\mathcal{N}\left(#1,#2\right)}

% Uniform distribution: arguments are the left and right endpoints
% E.g. $\unif{0}{1}$
\newcommand{\unif}[2]{\text{Uniform}(#1,#2)}

% Independent and identically distributed random variables
% E.g. $ X_1,...,X_n \iid \normal{0}{1}$
\newcommand{\iid}{\stackrel{\smash{\text{iid}}}{\sim}}

% Sequences (this shortcut is mostly to reduce finger strain for small hands)
% E.g. to write $\{A_n\}_{n\geq 1}$, do $\bk{A_n}{n\geq 1}$
\newcommand{\bk}[2]{\{#1\}_{#2}}

% Math mode symbols for common sets and spaces. Example usage: $\R$
\newcommand{\R}{\mathbb{R}}	% Real numbers
\newcommand{\C}{\mathbb{C}}	% Complex numbers
\newcommand{\Q}{\mathbb{Q}}	% Rational numbers
\newcommand{\Z}{\mathbb{Z}}	% Integers
\newcommand{\N}{\mathbb{N}}	% Natural numbers
\newcommand{\F}{\mathcal{F}}	% Calligraphic F for a sigma algebra
\newcommand{\El}{\mathcal{L}}	% Calligraphic L, e.g. for L^p spaces

% Math mode symbols for probability
\newcommand{\pr}{\mathbb{P}}	% Probability measure
\newcommand{\E}{\mathbb{E}}	% Expectation, e.g. $\E(X)$
\newcommand{\var}{\text{Var}}	% Variance, e.g. $\var(X)$
\newcommand{\cov}{\text{Cov}}	% Covariance, e.g. $\cov(X,Y)$
\newcommand{\corr}{\text{Corr}}	% Correlation, e.g. $\corr(X,Y)$
\newcommand{\B}{\mathcal{B}}	% Borel sigma-algebra

% Other miscellaneous symbols
\newcommand{\tth}{\text{th}}	% Non-italicized 'th', e.g. $n^\tth$
\newcommand{\Oh}{\mathcal{O}}	% Big-O notation, e.g. $\O(n)$
\newcommand{\1}{\mathds{1}}	% Indicator function, e.g. $\1_A$

% Additional commands for math mode
\DeclareMathOperator*{\argmax}{argmax}		% Argmax, e.g. $\argmax_{x\in[0,1]} f(x)$
\DeclareMathOperator*{\argmin}{argmin}		% Argmin, e.g. $\argmin_{x\in[0,1]} f(x)$
\DeclareMathOperator*{\spann}{Span}		% Span, e.g. $\spann\{X_1,...,X_n\}$
\DeclareMathOperator*{\bias}{Bias}		% Bias, e.g. $\bias(\hat\theta)$
\DeclareMathOperator*{\ran}{ran}			% Range of an operator, e.g. $\ran(T) 
\DeclareMathOperator*{\dv}{d\!}			% Non-italicized 'with respect to', e.g. $\int f(x) \dv x$
\DeclareMathOperator*{\diag}{diag}		% Diagonal of a matrix, e.g. $\diag(M)$
\DeclareMathOperator*{\trace}{trace}		% Trace of a matrix, e.g. $\trace(M)$
\DeclareMathOperator*{\supp}{supp}		% Support of a function, e.g., $\supp(f)$

% Numbered theorem, lemma, etc. settings - e.g., a definition, lemma, and theorem appearing in that 
% order in Lecture 2 will be numbered Definition 2.1, Lemma 2.2, Theorem 2.3. 
% Example usage: \begin{theorem}[Name of theorem] Theorem statement \end{theorem}
\theoremstyle{definition}
\newtheorem{theorem}{Theorem}[section]
\newtheorem{proposition}[theorem]{Proposition}
\newtheorem{lemma}[theorem]{Lemma}
\newtheorem{corollary}[theorem]{Corollary}
\newtheorem{definition}[theorem]{Definition}
\newtheorem{example}[theorem]{Example}
\newtheorem{remark}[theorem]{Remark}

% Un-numbered theorem, lemma, etc. settings
% Example usage: \begin{lemma*}[Name of lemma] Lemma statement \end{lemma*}
\newtheorem*{theorem*}{Theorem}
\newtheorem*{proposition*}{Proposition}
\newtheorem*{lemma*}{Lemma}
\newtheorem*{corollary*}{Corollary}
\newtheorem*{definition*}{Definition}
\newtheorem*{example*}{Example}
\newtheorem*{remark*}{Remark}
\newtheorem*{claim}{Claim}

% --- Left/right header text (to appear on every page) ---

% Do not include a line under header or above footer
\pagestyle{fancy}
\renewcommand{\footrulewidth}{0pt}
\renewcommand{\headrulewidth}{0pt}

% Right header text: Lecture number and title
\renewcommand{\sectionmark}[1]{\markright{#1} }
\fancyhead[R]{\small\textit{\nouppercase{\rightmark}}}

% Left header text: Short course title, hyperlinked to table of contents
\fancyhead[L]{\hyperref[sec:contents]{\small CS 170 Notes}}

% --- Document starts here ---

\begin{document}

% --- Main title and subtitle ---

\title{CS 170 Course Notes \\[1em]
\normalsize Spring 2023}

% --- Author and date of last update ---

\author{\normalsize Justin Wu, Brandon Wong}
\date{\normalsize\vspace{-1ex} Last updated: \today}

% --- Add title and table of contents ---

\maketitle
\tableofcontents\label{sec:contents}

% --- Main content: import lectures as subfiles ---

\newpage

% TeX root = ../Main.tex

% First argument to \section is the title that will go in the table of contents. Second argument is the title that will be printed on the page.
% \section[Lecture 6 SP 20 -- {\it DFS, Find SCC, BFS}]{Lecture 6 SP 20}
\section[Divide and Conquer -- {\it First Lectures SP23}]{Divide and Conquer}

\subsection{Algorithms: }

Fast Fourier Transform (FFT):

\begin{enumerate}
    \item Use Roots of Unity: $e^{i \theta}$ where $\theta = 2 \pi / 2$ is the angle on the complex plane
    \item Usually $n$ is some power of $2$ (can evenly divide the complex circle)
    \item Good for Convolutions (Pattern matching)
\end{enumerate}

\subsection{Notable Concepts}

\begin{enumerate}
    \item Usually recurse on two subproblems, then merge back
    \item Proofs often follow induction (divided groups will fall under the hypothesis)
    \item Runtime Strategies: (1) Master theorem, (2) Squeeze, (3) Guess and Check
    
\end{enumerate}



% TeX root = ../Main.tex

% First argument to \section is the title that will go in the table of contents. Second argument is the title that will be printed on the page.
% \section[Lecture 6 SP 20 -- {\it DFS, Find SCC, BFS}]{Lecture 6 SP 20}
\section[DFS, Find SCC, BFS -- {\it Lecture 6}]{DFS, Find SCC, BFS}


\subsection{Algorithms: }
    \begin{enumerate}
        \item \textbf{Finding SCC}:
        \begin{verbatim}
        deduce G^R from G
        Run DFS on G^R to have post numbers
        For v in revesre post order of G^R:
            if not visited[V]:
                explore(G, v)
                SCC++
        \end{verbatim}
            \begin{itemize}
                \item Main idea: Run explore() on some node in sink SCC.
                \item Intuition: When we run DFS on some sink node, we can trace back the entire SCC. It does not work the other way around since running DFS on source node goes all over the place
                \item Get $G^R$ (G but with all edges in reverse order). Run DFS on $G^R$ to get the post numbers of all vertices. This is to get all the sink nodes equivalents in the original graph in a particular order. For all vertices in $G^R$ in reverse post number, run explore(G, V). When we done exploring from that node, we know we have identified one SCC.
            \end{itemize}
        \item \textbf{BFS}:
        \begin{verbatim}
        def BFS(G, s)
            dist[s] = 0
            dist[V-S]=inf
            queue = [s]
            while q:
                for (u,v) in E:
                    if dist[v] = inf: # if v is unvisited
                        queue.add(v)
                        dist[v] = dist[u] + 1                    
        \end{verbatim}
            \begin{itemize}
                \item Runtime: V + E since we are seeing all vertices and edges only once
            \end{itemize}
    \end{enumerate}

\subsection{Notable Concepts}

\begin{enumerate}
    \item Strong Connected Graph: any pair of vertices can reach each other forwards and backwards

    \item A strongly connected graph can have only one strongly connected component, whereas a connected directed graph can have multiple strongly connected components

    \item Every digraph G is a DAG on SCCs
        \begin{itemize}
            \item Otherwise, it would shrink back into a super-strongly connected component
        \end{itemize}
    \item DFS, and types of directed edges: (when keeping track of DFS pre/post order times)
        \begin{itemize}
            \item Tree edges and Forward edges: The beginning vertex has an interval that contains the interval of the destination vertex.
            \item Back Edges: The destination vertex has an interval that contains the interval of the beginning vertex.
            \item Cross Edges: The intervals are disjoint.
        \end{itemize}

\end{enumerate}






% TeX root = ../Main.tex

% First argument to \section is the title that will go in the table of contents. Second argument is the title that will be printed on the page.
% \section[Lecture 13 (2/13) -- {\it Paths in Graphs}]{Lecture 13 (2/13)}
\section[Dijkstras, Bellman-Ford -- {\it Lecture 7}]{Dijkstras, Bellman-Ford}



\subsection{Algorithms}

    \begin{enumerate}
        \item \textbf{Dijkstra}
        \begin{verbatim}
            asdf
        \end{verbatim}
        \item \textbf{Bellman-Ford}: Visualize the algorithm as running through a table. Outer loop takes down the rows (vertices), inner loop takes down the columns (edges). Then we argue that the table contains all the possible paths, and so it will contain the shortest path.
            \begin{verbatim}
            dist[s]=0, dist[v-s]=inf
            for i in |v|-1:
                for all edges:
                    update(e)
            \end{verbatim}

            \begin{itemize}
                \item Runtime: $O(|V|*|E|)$
                \item Dijkstra is basically a trimmed down BF since we are selective choosing a very particular order of updates so we will do fewer updates in total, using the priority queue as a guide to choose what's the next edge to update, whereas BF goes for all the paths.
            \end{itemize}
        
    \end{enumerate}



\subsection{Notable Concepts}

    \begin{enumerate}
        \item Proof of Correctness for Dijkstra
            \begin{itemize}
                \item Base case: subset $K$ of $V$ where we know the shortest path from $s$ to all $v\in K$
                \item Inductive step: To prove the statement that, if a vertex $A$ is the nearest vertex right outside the boundary of $K$, then we also know its shortest path. Prove use contradiction by assuming that there exist another vertex $B$ between boundary of $K$ and $A$ that has the shortest distance, but we know that is not possible given that we know $A$ already has the shorte
            \end{itemize}
        \item Important properties of the update subroutine of Dijkstra:
            \begin{itemize}
                \item update(u, v) $\equiv$ dist[v] = min(dist[v], dist[u] + length(u, v))
                \item update is "safe": provided that dist[u] is at least the correct answer (dist(s, u)) AND dist[v] is also at least dist(s,v), the the updated dist[v] is as good as the correct distance dist(s,v)
                \item If we know a path from s to v that is currently known as the shortest path and that dist[u] is correct, then calling update[u, v] makes dist[v] correct.
            \end{itemize}
        \item When talking about negative edges, we care about graphs without negative cycles.
        \item Dijkstra's correctness relies on the fact that the further you go down the path, the greater the truth. But this fact is not true anymore when you have negative edges, since an immediate edge with expensive weights can have edges with negative weights after it that makes the entire path the shortest.

        
        
    \end{enumerate}






\section[Negative Cycle, DAG shortest path, Greedy -- {\it Lecture 8}]{Negative Cycle, DAG shortest path, Greedy}

\subsection{Algorithms: }
    \begin{enumerate}
        \item \textbf{Detecting negative cycle}: Run one additional iteration of Bellman-Ford after the algorithm finishes running. If there's change in the distance of shortest path, then there is a negative cycle.
        \begin{verbatim}
        Run Bellman-Ford
        for all edges:
            if update(e) modifies distance of endpoints:
                halt and output "NEG cylce"
        \end{verbatim}

        \begin{itemize}
            \item Proof:
            
            Correctness of Bellman-Ford suggests that if there are no negative cycles $\rightarrow$ nothing changes.
            
            Now we want to show that if nothing changes $\rightarrow$ no negative cycles.

            Consider a graph with two vertices $a, b$. There is a cycle where $a$ points to $b$ and vice versa. We want to show that length($a, b$) + length($b, a$) $\geq$0. Given that nothing changes, we know that dist[b]$\leq$dist[a]+length(a,b) and dist[a]$\leq$dist[b]+length(b,a). If we add the inequalities up, we find that length(a,b)+length(b,a) is indeed $\geq$ 0.

            \item Runtime:

            Same as Bellman-Ford: $O(V\cdot E)$
            
        \end{itemize}

        \item \textbf{Finding shortest path in DAG}:
        \begin{verbatim}
        Topologically sort the graph
        for all v in topo order:
            for all (v,w): # for all neighbors of v
                update(v,w)
        \end{verbatim}

        \begin{itemize}
            \item Runtime: $O(V+E)$ since sorting takes $V$ runtime and we check each edge once.
        \end{itemize}

        \item \textbf{Largest of non-overlapping intervals: }
        \begin{verbatim}
        sort(intervals) in ascending endtimes
        Choose the earliest interval, put it in set S
        for all interval in (intervals - S):
            if interval does not overlap last interval in S:
                S.add(interval)
        return len(S)
        \end{verbatim}
        \begin{itemize}
            \item Proof of Correctness: Suppose greedy solution $G$ is not optimal. Pick an optimal solution $S$, where $S$ has the most overlapping with G ($max(G\cap S)$). Now consider a new solution $S'$ with same entries as $S$. Say $S'$ comprises of intervals $S_1, S_2, \cdots, S_n$. Now we go to first point in $S'$ where the interval starts differing, $S_i$, and we replace it with the corresponding interval in $G_i$. $G_i$ fits in with the previous intervals because $G$ was part of the solution of $G$. Moreover, by definition of the greedy algorithm, $G_i$ will also be the interval with the earliest end time after all previous intervals, thus $G_i$ can fit in the original interval $S_i$ that it replaces. Perform the same action to the remaining differing intervals. After replacement, $S'$ will have more intersection with $G$ than $S$ from all the new intervals from $G$, which contradicts with our prior claim that $S$ has the highest resemblance to $G$.
        \end{itemize}

        \item \textbf{Set Cover: }
        \begin{verbatim}
        J = {}
        while unions of sets in J != U:
            pick a set not in J that adds the most new elements
            add that set to J
        return J
        \end{verbatim}
        \begin{itemize}
            \item Define $n_t$ as the number of uncovered points after $t$ iterations. Example: $n_0=|U|, n_1=...$. Lemma: $n_{t+1}\leq n_t - \frac{n_t}{|J*|}$. Since $|J*|$ is the optimal solution, the minimum number of subsets that covers $U$ is $|J*|$. Therefore, the size of the next subset chosen by the algorithm in $n_t$ has to be at least $\frac{n_t}{|J*|}$ since by definition, the algorithm chooses the subset that can add the most points.
            
            \item Proof that this algorithm outputs a solution $|J|$ that's within a logarithmic coefficient to the real optimal solution $|J*|$ involves algebraic manipulation of the lemma defined above.
        \end{itemize}
    \end{enumerate}


\subsection{Notable Concepts}

\begin{enumerate}
    \item Summary: Which shortest path algorithm to use?
        \begin{itemize}
            \item Any graph (cycles or not) and lengths > 0 $\rightarrow$ Dijkstra
            \item Graph w/o negative cycles $\rightarrow$ BF (can in fact detect)
        \end{itemize}
    \item There is not a "fully optimal" algorithm that runs in polynomial time for set cover, though the existing algorithm is at the logarithmic coefficient to the optimal solution.
\end{enumerate}




% --- Bibliography ---

% Start a bibliography with one item.
% Citation example: "\cite{williams}".

% \begin{thebibliography}{1}

% \bibitem{williams}
%    Williams, David.
%    \textit{Probability with Martingales}.
%    Cambridge University Press, 1991.
%    Print.

% Uncomment the following lines to include a webpage
% \bibitem{webpage1}
%   LastName, FirstName. ``Webpage Title''.
%   WebsiteName, OrganizationName.
%   Online; accessed Month Date, Year.\\
%   \texttt{www.URLhere.com}

% \end{thebibliography}

% --- Document ends here ---

\end{document}
